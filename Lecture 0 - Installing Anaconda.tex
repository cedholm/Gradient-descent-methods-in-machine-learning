\documentclass[addpoints]{exam}

% Packages
\usepackage{amsmath, amssymb, verbatim}
\usepackage[margin=1in]{geometry}
\usepackage{color}
\usepackage{float}
\usepackage{graphicx}
\usepackage[colorlinks=true]{hyperref}

% Style
\setlength{\parindent}{0pt}
\pagestyle{headandfoot}
\cfoot{ Page \thepage\,of \numpages}

% Define commands
\def\course{Gradient descent methods in machine learning}
\def\docTitle{Python Startup Guide}
\def\profName{Christina J. Edholm, Maryann E. Hohn, Ami E. Radunskaya}

% Shortcuts
\newcommand{\abs}[1]{\left \lvert#1\right \rvert} %absolute value

\begin{document}

% Heading
{\center \textsc{\Large \docTitle}\\
%	\vspace*{1em}
	\large\course\\
	\profName\\
	\vspace*{1em}
	\hrule
\vspace*{2em}}

We recommend downloading the latest version of Python 3 on your computer if it is not there already.

\section{Downloading Jupyter Notebook}

\textbf{If Python is not already downloaded on your computer}, one of the easiest ways to download and manage packages in Python is by using Anaconda at
\begin{center}
    \href{https://www.anaconda.com/download/}{https://www.anaconda.com/download/}.
\end{center}

The site contains the newest Python distribution and the ability to use Jupyter Notebook.\\

\textbf{If Python is already downloaded on your computer}, be sure to update your packages. If you do not have Jupyter Notebook installed, it can be downloaded separately at
\begin{center}
    \href{https://jupyter.org/install}{https://jupyter.org/install}.
\end{center}

\section{Getting started}
Once Jupyter Notebook is installed, you can start the program in a variety of ways.  In all cases, your default web browser will launch the program, which will start with a list of files on your computer.  Here are the top two:
\begin{itemize}
    \item If you downloaded Anaconda, click on Anaconda Navigator in your applications folder.  A list of programs appears where one of the programs is Jupyter Notebook.  Click on Jupyter Notebook.  
    %\item If you did not download Anaconda, find Jupyter Notebook on your computer.
    \item Open Terminal (Mac) or the Command Prompt (Windows), depending on your computer.  Type 
    \begin{verbatim}
        jupyter notebook
    \end{verbatim}
\end{itemize}
To get started, we highly suggest reading the tutorial on Jupyter Notebook at
\begin{center}
    \href{https://www.datacamp.com/community/tutorials/tutorial-jupyter-notebook}{https://www.datacamp.com/community/tutorials/tutorial-jupyter-notebook}.
\end{center}
Another great way to get started is to read/do the examples at
\begin{center}
    \href{http://hplgit.github.io/bumpy/doc/pub/._basics000.html}{http://hplgit.github.io/bumpy/doc/pub/.\_basics000.html}.
\end{center}
They include how to evaluate mathematical functions, how to print, for loops, while loops, lists, arrays, and using packages like numpy, matplotlib.pyplot, and math.  The examples are great! Highly recommended.\\

If you are a video person and not familiar with Python, you may want to consider watching some DataCamp videos
\begin{center}
    \href{https://www.datacamp.com/courses/intro-to-python-for-data-science}{https://www.datacamp.com/courses/intro-to-python-for-data-science}
\end{center}

\end{document}
